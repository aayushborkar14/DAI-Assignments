\section{A Pretty ``Normal'' Mixture}
\subsection*{Task A}
Let the output of the algorithm is $f_{\cal{A}}$, which is to be determined.
First we define an event as below.
\[A_i\text{ : Random Variable }X_i\text{ is chosen}\]
According to the given algorithm, we first select a Gaussian variable $X_i$ with probability $P[A_i]=p_i$, and then sample from $X_i$.
Once any $X_i$ is selected, we sample from that $X_i$, with probability $P[X_i = x]$ (i.e., the PDF of gaussian variable $X_i$). From this, it directly follows that $P[X=x\mid A_i] = P[X_i = x]$.
Using law of total probability, we write
\begin{align*}
	P[X=u]                             & = \sum_{i=1}^K P[X=u\mid A_i]P[A_i] \\
	\therefore f_{\cal{A}}(u) = P[X=u] & = \sum_{i=1}^K p_i P[X_i=u]
\end{align*}
The above expression is equal to $f_X(u)$.

\subsection*{Task B}
\begin{remark}
	We know that the following properties of a Gaussian random variable $X_i \sim \mathcal{N}(\mu_i, \sigma_i^2)$:
	\begin{align}
		\EX[X_i]   & = \mu_i \label{eqn:E1}                                                               \\
		\Var[X_i]  & = \sigma_i^2 \label{eqn:V1}                                                          \\
		M_{X_i}(t) & = e^{\mu_i t + \frac{\sigma_i^2 t^2}{2}} \label{eqn:M1}                              \\
		f_{X_i}(u) & = \frac{1}{\sqrt{2\pi\sigma_i^2}}e^{-\frac{(u-\mu_i)^2}{2\sigma_i^2}} \label{eqn:F1}
	\end{align}
	From $\Var[X_i]$, we can also calculate $\EX[X_i^2]$ as
	\begin{align}
		\Var[X_i]  & = \EX[X_i^2] - \EX[X_i]^2 \nonumber \\
		\EX[X_i^2] & = \Var[X_i] + \EX[X_i]^2 \nonumber  \\
		\EX[X_i^2] & = \sigma_i^2 + \mu_i^2
	\end{align}
\end{remark}

\subsubsection*{1. $\EX[X]$}
\begin{align*}
	\EX[X] & = \int_{-\infty}^{\infty} x f_X(x) \, dx                    \\
	       & = \int_{-\infty}^{\infty} x \sum_{i=1}^K p_i P[X_i=x] \, dx \\
	       & = \sum_{i=1}^K p_i \int_{-\infty}^{\infty} x P[X_i=x] \, dx \\
	       & = \sum_{i=1}^K p_i \EX[X_i]
\end{align*}
From \cref{eqn:E1}, we have
\begin{align}
	\therefore \EX[X] & = \sum_{i=1}^K p_i \mu_i \label{eqn:E2}
\end{align}
\subsubsection*{2. $\Var[X]$}
To calculate $\Var[X]$, we first calculate $\EX[X^2]$.
\begin{align*}
	\EX[X^2] & = \int_{-\infty}^{\infty} x^2 f_X(x) \, dx                    \\
	         & = \int_{-\infty}^{\infty} x^2 \sum_{i=1}^K p_i P[X_i=x] \, dx \\
	         & = \sum_{i=1}^K p_i \int_{-\infty}^{\infty} x^2 P[X_i=x] \, dx \\
	         & = \sum_{i=1}^K p_i \EX[X_i^2]                                 \\
\end{align*}

From \cref{eqn:V1}, we get
\begin{align}
	\EX[X^2] & = \sum_{i=1}^K p_i (\sigma_i^2 + \mu_i^2) \label{eqn:V2}
\end{align}
And then, we proceed to calculate $\Var[X]$ using \cref{eqn:E2} and \cref{eqn:V2}.
\begin{align}
	\Var[X]            & = \EX[X^2] - \EX[X]^2                                                             \nonumber      \\
	\therefore \Var[X] & = \sum_{i=1}^K p_i (\sigma_i^2 + \mu_i^2) - \left(\sum_{i=1}^K p_i \mu_i\right)^2 \label{eqn:V3}
\end{align}

\subsubsection*{3. \normalfont the MGF $M_X(t)$ of $X$}
\begin{align*}
	M_X(t) & = \EX[e^{tX}]                                                    \\
	       & = \int_{-\infty}^{\infty} e^{tx} f_X(x) \, dx                    \\
	       & = \int_{-\infty}^{\infty} e^{tx} \sum_{i=1}^K p_i P[X_i=x] \, dx \\
	       & = \sum_{i=1}^K p_i \int_{-\infty}^{\infty} e^{tx} P[X_i=x] \, dx \\
	       & = \sum_{i=1}^K p_i M_{X_i}(t)
\end{align*}
From \cref{eqn:M1}, we get
\begin{align}
	\therefore M_X(t) & = \sum_{i=1}^K p_i e^{\mu_i t + \frac{\sigma_i^2 t^2}{2}} \label{eqn:M2}
\end{align}

\subsection*{Task C}
For this task, we are given a random variable $Z$ as follows,
\[Z = \sum_{i=1}^K p_i X_i\]

\subsubsection*{1. $\EX[X]$}
This can be calculated using linearity of expectation.
\begin{align}
	\EX[Z]            & = \sum_{i=1}^K p_i \EX[X_i] \nonumber  \\
	\therefore \EX[Z] & = \sum_{i=1}^K p_i \mu_i \label{eq:E4}
\end{align}

\subsubsection*{4. \normalfont the MGF $M_Z(t)$ of $Z$}
To simplify the calculation of Variance, we calculate the Moment Generating Function of $Z$ first.
\begin{align}
	M_Z(t) & = \EX[e^{tZ}] \nonumber
	= \EX[e^{t\sum_{i=1}^K p_i X_i}] \nonumber                                                                          \\
	       & = \EX[\prod_{i=1}^K e^{tp_i X_i}] \nonumber
	= \prod_{i=1}^K \EX[e^{tp_i X_i}] \nonumber                                                                         \\
	       & = \prod_{i=1}^K M_{X_i}(tp_i) \nonumber
	= \prod_{i=1}^K e^{tp_i \mu_i + \frac{t^2 p_i^2 \sigma_i^2}{2}}\nonumber                                            \\
	       & = \exp\left({\sum_{i=1}^K p_i \mu_i t + \frac{t^2 \sum_{i=1}^K p_i^2 \sigma_i^2}{2}} \right)\label{eqn:M3}
\end{align}

\subsubsection*{2. $\Var[Z]$}
The variance is the second derivative of the MGF at $t=0$.
We first calculate the first and second derivatives of $M_Z(t)$, from \cref{eqn:M3}.
\begin{align}
	\diff{M_Z(t)}{t}    & = \left(\sum_{i=1}^K p_i \mu_i + t \sum_{i=1}^K p_i^2 \sigma_i^2 \nonumber\right) M_Z(t)                                                         \\
	\diff[2]{M_Z(t)}{t} & = \left(\sum_{i=1}^K p_i^2 \sigma_i^2\right) M_Z(t) + \left(\sum_{i=1}^K p_i \mu_i + t\sum_{i=1}^Kp_i^2\sigma_i^2\right)^2 M_Z(t) \label{eqn:M4}
\end{align}
Now, substituting $t=0$ in \cref{eqn:M4}, we get $\EX[Z^2]$, and finally $\Var[Z]$.
\begin{align}
	\EX[Z^2]           & = \sum_{i=1}^K p_i^2 \sigma_i^2 + \left(\sum_{i=1}^K p_i \mu_i\right)^2 \nonumber                                         \\
	\Var[Z]            & = \EX[Z^2] - \EX[Z]^2 \nonumber                                                                                           \\
	                   & = \sum_{i=1}^K p_i^2 \sigma_i^2 + \left(\sum_{i=1}^K p_i \mu_i\right)^2 - \left(\sum_{i=1}^K p_i \mu_i\right)^2 \nonumber \\
	\therefore \Var[Z] & = \sum_{i=1}^K p_i^2 \sigma_i^2 \label{eqn:V4}
\end{align}

\subsubsection*{3. \normalfont the PDF $f_Z(u)$ of $Z$}
Observe that the MGF of $Z$ can be written as
\begin{align*}
	M_Z(t)                                        & = \exp\left(\mu t + \frac{t^2\sigma^2}{2} \right)       \\
	\textrm{Where, } \mu = \sum_{i=1}^K p_i \mu_i & \textrm{ and } \sigma^2 = \sum_{i=1}^K p_i^2 \sigma_i^2
\end{align*}
This MGF is the same as that of a Gaussian random variable, from \cref{eqn:M1}.
We state the following theorem, with proof for finite discrete case in \hyperref[taskD]{Task D}.
\begin{theorem}\label{thm:MGF-PDF}
	For a random variable $X$, if it is
	\begin{enumerate}
		\item either finite and discrete,
		\item or if it is continuous and its MGF $\phi_X(t)$ is known for some (non-infinitesimal) interval,
	\end{enumerate}
	then its MGF and PDF \textbf{uniquely} determine each other.
\end{theorem}
Knowing that $Z$ has a Gaussian MGF and hence a Gaussian PDF, we can conclude that $Z$ is a Gaussian random variable.
And from \cref{eqn:F1}, we can find the PDF of $Z$.
\begin{align}
	Z                 & \sim \mathcal{N}\left(\sum_{i=1}^K p_i\mu_i, \sum_{i=1}^K p_i^2\sigma_i^2 \right)                                                                    \label{eq:gaussian-dist} \\
	\therefore f_z(u) & = \frac{1}{\sqrt{2\pi \sum_{i=1}^K p_i^2 \sigma_i^2}} \exp\left(-\frac{\left(u-\sum_{i=1}^K p_i\mu_i\right)^2}{2\sum_{i=1}^K p_i^2 \sigma_i^2}\right) \label{eq:P4}
\end{align}

\subsubsection*{5. \normalfont What can you conclude? Do $X$ and $Z$ have the same properties?}
No, $X$ and $Z$ have different properties. Everything except the mean is different for both.

\subsubsection*{6. \normalfont What distribution does $Z$ seem to follow?}
$Z$ follows a Gaussian distribution, as seen in \cref{eq:gaussian-dist}.

\subsection*{Task D}\label{taskD}
In this task, we prove \cref{thm:MGF-PDF} for the finite discrete case.
\begin{remark}
	For a discrete random variable, we consider PMF instead of PDF, and use $f_X(u)$ to denote the PMF of $X$.
\end{remark}
Consider two random variables $X:\Omega \to \mathbb{R}$ and $Y:\Omega\to\mathbb{R}$, with $\Omega = \{\omega_1, \omega_2,\cdots, \omega_k\}$.
The MGF of $X$ can be written as
\begin{align*}
	M_X(t) & = \EX[e^{tX}]                              \\
	       & = \sum_{i=1}^k e^{t\omega_i} f_X(\omega_i) \\
\end{align*}

We prove the theorem in both directions, i.e.,
\begin{enumerate}
	\item If the PDF's are equal, then the MGF's are equal.
	\item If the MGF's are equal, then the PDF's are equal.
\end{enumerate}

\subsubsection*{Proving that $f_X=f_Y \implies M_X = M_Y$}
Given that $f_X(u)=f_Y(u)$ $\forall u\in\mathbb{R}$, we can directly compute $M_X$ and $M_Y$ and show that they are equal.
\begin{align*}
	M_X(t)         & = \sum_{i=1}^k e^{t\omega_i} f_X\left(\omega_i\right) \hspace{1em}\forall t\in\mathbb{R}
	= \sum_{i=1}^k e^{t\omega_i} f_Y\left(\omega_i\right)                                                     \\
	M_X(t)         & = M_Y(t) \hspace{1em} \forall t\in\mathbb{R}                                             \\
	\therefore M_X & = M_Y
\end{align*}

\subsubsection*{Proving that $M_X=M_Y \implies f_X = f_Y$}
Given that $M_X(t) = M_Y(t)$ $\forall t\in \mathbb{R}$, we have the following equation.
\begin{align*}
	M_X(t)                                                                 & = M_Y(t)                                   \\
	\sum_{i=1}^k e^{t\omega_i} f_X(\omega_i)                               & = \sum_{i=1}^k e^{t\omega_i} f_Y(\omega_i) \\
	\sum_{i=1}^k e^{t\omega_i} \left(f_X(\omega_i) -  f_Y(\omega_i)\right) & = 0
\end{align*}
We introduce the following notation to simplify the calculations.
\begin{align*}
	z                                      & = e^t  \implies z > 0           \\
	c_i                                    & = f_X(\omega_i) - f_Y(\omega_i) \\
	\implies \sum_{i=1}^k z^{\omega_i} c_i & = 0\hspace{1em}\forall z>0
\end{align*}

\begin{claim}\label{claim:coeff-zero}
	All $c_i$'s are zero.
\end{claim}
\begin{proof}
	Assume, for the sake of contradiction, that there exists some $c_i \neq 0$. Let $c_m$ be such a non-zero $c_i$ with the smallest $i$, i.e., $m = \min\{i: c_i \neq 0\}$.
	\begin{align}
		\sum_{i=1}^k z^{\omega_i} c_i & = 0   \hspace{1em} \forall z > 0            \nonumber           \\
		\sum_{i=m}^k z^{\omega_i} c_i & = 0\textrm{ as } c_i = 0 \textrm{ for } i < m \label{eq:coeff1}
	\end{align}
	Now, we divide the \cref{eq:coeff1} by $z^{\omega_m}$. We get
	\begin{align}
		\sum_{i=m+1}^k z^{\omega_i - \omega_m} c_i + c_m & = 0\label{eq:coeff2}
	\end{align}
	Taking limit as $z \to 0^+$ on both sides of \cref{eq:coeff2}, we get
	\begin{align*}
		\lim_{z\to 0^+} \sum_{i=m+1}^k z^{\omega_i - \omega_m} c_i + c_m & = 0 \\
		c_m                                                              & = 0
	\end{align*}
	$\therefore$ A contradiction. Hence, all $c_i$'s are zero.
\end{proof}
Following claim \ref{claim:coeff-zero}, we have $c_i = 0$ $\forall i$.
\begin{align*}
	c_i = f_X(\omega_i) - f_Y(\omega_i) & = 0 \hspace{1em} \forall i                                   \\
	f_X(\omega_i)                       & = f_Y(\omega_i) \hspace{1em} \forall i                       \\
	\textrm{Also, }	f_X(u) = 0          & \textrm{ and } f_Y(u)=0 \hspace{1em} \forall u \notin \Omega \\
	f_X(u)                              & = f_Y(u) \hspace{1em}\forall u\in \mathbb{R}                 \\
	\therefore	f_X                      & = f_Y
\end{align*}
This completes out proof for \cref{thm:MGF-PDF} in the finite discrete case.
