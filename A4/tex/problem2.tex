\section{Forecasting on a Real World Dataset.}
\subsubsection{Predicting \texttt{PASSENGERS CARRIED}}
The code along with explanation is given in \texttt{2-1a.ipynb}, \texttt{2-1b.ipynb} and \texttt{2-1c.ipynb} respectively.

\subsubsection{Why MAPE is not the best metric?}
MAPE may not be the best metric to evaluate the forecasts because:
\[\text{MAPE} = \frac{1}{n} \sum_{i=1}^{n} \left| \frac{y_i - \hat{y}_i}{y_i} \right| \times 100\]
\begin{itemize}
	\item Gives misleading results when the data has close to zero values, as the denominator is very small.
	\item Doesn't capture the assymetric cost of errors, i.e., positive and negative errors are treated equally while in reality one results in understaffing, while the other results in overstaffing.
\end{itemize}

For fleet management, we focus on the total number of passengers carried. Instead of MAPE, we can use RMSE
\[\text{RMSE} = \sqrt{\frac{1}{n} \sum_{i=1}^{n} (y_i - \hat{y}_i)^2}\]
For human resources, we focus on the peak demand. In this case, we can use peak weighted RMSE
\[ \text{Peak weighted RMSE} = \sqrt{\frac{1}{n} \sum_{i=1}^{n} w_i (y_i - \hat{y}_i)^2}\]
where $w_i$ is the weight of the $i$-th observation.

\subsubsection{Testing if mean is different pre and post COVID}
To test if the mean, $\mu$, differs between the pre-COVID and post-COVID periods,
we can perform a \textbf{two-sample t-test}. This test compares the means of two independent groups (in this case, the pre-COVID period before December 2019 and the post-COVID period after January 2022) to determine if there is a significant difference in μ between them.
Since we assume the data is weakly stationary and normally distributed with a known variance,
this test would be appropriate to assess if there is a change in the mean differenced series across these two periods.
