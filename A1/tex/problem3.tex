\section{Random Variables}
\subsection{}
We are given the following inequalities:
\[
	P(Q_1 < q_1) \geq 1 - p_1 \quad \text{and} \quad P(Q_2 < q_2) \geq 1 - p_2
\]
We check the complementary probabilities for the events \( Q_1 \geq q_1 \) and \( Q_2 \geq q_2 \), we get the following inequalities:
\[
	P(Q_1 \geq q_1) = P((Q_1 < q_1)^c) = 1 - P(Q_1 < q_1)
\]
Given that \( P(Q_1 < q_1) \geq 1 - p_1 \), we see that:
\[
	P(Q_1 \geq q_1) \leq p_1
\]
Similarly, for \( Q_2 \), we have:
\[
	P(Q_2 \geq q_2) \leq p_2
\]
We use the probability of the union of events \( Q_1 \geq q_1 \) and \( Q_2 \geq q_2 \):
\[
	P[(Q_1 \geq q_1) \cup (Q_2 \geq q_2)] + P[(Q_1 \geq q_1) \cap (Q_2 \geq q_2)] = P(Q_1 \geq q_1) + P(Q_2 \geq q_2)
\]
Since \( P(Q_1 \geq q_1) \leq p_1 \) and \( P(Q_2 \geq q_2) \leq p_2 \), we obtain:
\[
	P(Q_1 \geq q_1) + P(Q_2 \geq q_2) \leq p_1 + p_2
\]
\[
	P[(Q_1 \geq q_1) \cup (Q_2 \geq q_2)] + P[(Q_1 \geq q_1) \cap (Q_2 \geq q_2)] \leq p_1 + p_2
\]
\[
	P[(Q_1 \geq q_1) \cup (Q_2 \geq q_2)] \leq p_1 + p_2 - P[(Q_1 \geq q_1) \cap (Q_2 \geq q_2)]
\]
Using \textit{De Morgan’s Law},  \( (A \cup B)^c = A^c \cap B^c \), we work with the following complementary event:
\[
	P[(Q_1 < q_1) \cap (Q_2 < q_2)] = P[((Q_1 \geq q_1) \cup (Q_2 \geq q_2))^c] = 1 - P[(Q_1 \geq q_1) \cup (Q_2 \geq q_2)]
\]
Since \( P[(Q_1 \geq q_1) \cup (Q_2 \geq q_2)] \geq 0 \), we conclude:
\[
	P[(Q_1 < q_1) \cap (Q_2 < q_2)] \geq 1 - (p_1 + p_2)
\]
This gives a lower bound on the probability that both \( Q_1 \) and \( Q_2 \) are less than \( q_1 \) and \( q_2 \).
Claim:
\[
	(Q_1 < q_1) \cap (Q_2 < q_2) \subseteq \{ Q_1 Q_2 < q_1 q_2 \}
\]
Let \( x \in (Q_1 < q_1) \cap (Q_2 < q_2) \). This means:
\[
	Q_1(x) < q_1 \quad \text{and} \quad Q_2(x) < q_2
\]
Since both \( Q_1(x) \) and \( Q_2(x) \) are positive, their product is also positive:
\[
	0 < Q_1(x) Q_2(x) < q_1 q_2
\]
Thus, \( x \in \{ Q_1 Q_2 < q_1 q_2 \} \), proving the claim.
Using the fact that \( (Q_1 < q_1) \cap (Q_2 < q_2) \subseteq \{ Q_1 Q_2 < q_1 q_2 \} \), we conclude:
\[
	P(Q_1 Q_2 < q_1 q_2) \geq P[(Q_1 < q_1) \cap (Q_2 < q_2)] \geq 1 - (p_1 + p_2)
\]
\[
	P(Q_1 Q_2 < q_1 q_2) \geq 1 - (p_1 + p_2)
\]

\subsection{}
We start by using the definition for the variance of a sample.

\begin{align*}
	\sigma^2 & = \frac{\sum_{i=1}^{n}(x_i - \mu)^2}{n - 1}
\end{align*}

Since squares are always positive, we can conclude the following inequality.

\begin{align*}
	\forall i,\; (x_i - \mu)^2 & \leq \sum_{i=1}^{n}(x_i - \mu)^2
\end{align*}

From the definition, we see that the total sum of squared deviations is equal to \( \sigma^2(n - 1) \). Hence we can further refine our inequality for our singular data point.

\begin{align*}
	(x_i - \mu)^2 & \leq \sigma^2(n - 1)
\end{align*}

Since both sides of the inequality are positive, we can conclude that it must hold even after taking the square root since it is a monotonically increasing function.

\begin{align*}
	|x_i - \mu| & \leq \sigma\sqrt{n - 1}
\end{align*}

This gives us our required result.

Now we compare this with **Chebyshev's Inequality**, which is stated as follows:

\[
	P(|X - \mu| \geq k\sigma) \leq \frac{1}{k^2}
\]

As $n$ increases we see that the probability of the deviation being more than $\sigma\sqrt{n - 1}$ decreases as $\frac{1}{n - 1}$ which is true since the previous inequality implies that all deviations lie within $\sigma\sqrt{n - 1}$. Hence Chebyshev's inequality is more accurate for larger sample sizes.
